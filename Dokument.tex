% ==========================
%  LaTeX 2e - Dokument
%  Editor: Dragan Kozulovic
% ==========================
\documentclass[11pt,a4paper,fleqn,twoside]{report}
\usepackage[dvips,final]{graphicx}               %% epsfig 
\usepackage[utf8]{inputenc}                    %% ISO-Text 
\usepackage[T1]{fontenc}                         %% 
\usepackage[german]{babel}                       %% BABEL 
\usepackage{parskip}                             %% 
\usepackage{fancyhdr}                            %% 
\usepackage[hang,bf]{caption}                    %% 
\usepackage{color}                               %% Farben
\usepackage{multirow}                            %% 
%\usepackage{natbib}
\usepackage{float}
\usepackage{siunitx}

\usepackage{amsmath}
\usepackage{graphicx}
\usepackage[onehalfspacing]{setspace}
%\usepackage[flushleft]{threeparttable}
\usepackage{blindtext}
\usepackage{amssymb}
%\usepackage{esvect}
%\usepackage[breaklinks]{hyperref} 



% Einzubindende Dateien 
\includeonly{Titelseite,
             Letzte_Seite,
	     eid_erkl,
	     Nomenklatur}
\graphicspath{{Bilder/}}                         %% Pfad fuer einzubindende Graphiken


% Seiten-Layout
\oddsidemargin   0.0cm                           %% Anpassung DIN A4-Format (symmetrisch)
\evensidemargin  0.0cm                           %% Anpassung DIN A4-Format (symmetrisch)
\topmargin      -1.0cm         
\textheight     25.0cm
\textwidth      16.0cm
\pagestyle{fancy}
\renewcommand{\chaptermark}[1]{\markboth{\thechapter.\ #1}{}}
\renewcommand{\sectionmark}[1]{\markright{\thesection\ #1}}
\fancyhf{}                                %Clears all header and footer fields, in preparation.
\fancyhead[LE,RO]{\thepage}               %Displays the page number in bold in the header,
                                          % to the left on even pages and to the right on odd pages.
\fancyhead[RE]{\nouppercase{\leftmark}}   %Displays the upper-level (chapter) information---
                                          % as determined above---in non-upper case in the header, to the right on even pages.
\fancyhead[LO]{\rightmark}                %Displays the lower-level (section) information---as
                                          % determined above---in the header, to the left on odd pages.
\renewcommand{\headrulewidth}{0.5pt}      %Underlines the header. (Set to 0pt if not required).

%\sloppy                                         %% lockerer Zeilenumbruch
%\flushbottom                                    %% buendige letzte Zeile


% Trennungskorrekturen
%\hyphenation{Bei-spiel}
%\hyphenation{}
%\hyphenation{}
%\hyphenation{}
%\hyphenation{}
%\hyphenation{}


% Umbenennungen (babel) (siehe LaTeX-Begleiter, Abschn. 9.2.3)
\addto\extrasgerman{\renewcommand{\figurename}{Abb.}}


% Listen
\newcommand{\bl}{\begin{list}{\textbullet}%         %% kleine Aufzaehlung/Liste
{\topsep0pt\partopsep0pt\itemsep0pt\parsep0pt\leftmargin1.5em\labelwidth1em\labelsep0.5em}}
\newcommand{\el}{\end{list}}
\newcommand{\cit}[1]{\textit{\cite{#1}}}


% Literaturverzeichnis
\usepackage[backend=biber,						% Bibtex-Backend biber mit ACII und Latin 1 Unterst�tzung
bibencoding=auto,
natbib=true,									% natbib Zitierbefehle
bibstyle=numeric,								% Numerisches Literaturverzeichnis
giveninits=true,								% Vor- und Mittelnamen als Initialien
citestyle=numeric,								% Numrische Zitierweise
url=false,										% Schreibe nicht die URL ins Literaturverzeichnis
maxbibnames=3,									% Maximal 3 Autoren im Literaturverteichnis
minbibnames=3									% Maximal 3 Autoren im Literaturverteichnis
]{biblatex}
\bibliography{Literatur}		% Pfad zur *.bib-Datei
\DeclareNameAlias{default}{last-first}			% Nachname, Vorname in Literaturverzeichnis
\renewcommand{\labelnamepunct}{\addcolon\space}	% Doppelpunkt nach Autoren vor Titel
\renewbibmacro*{journal+issuetitle}{			% number in Klammern
  \usebibmacro{journal}%
  \setunit*{\addspace}%
  \iffieldundef{series}
    {}
    {\newunit
     \printfield{series}%
     \setunit{\addspace}}%
  \printfield{volume}%
  \iffieldundef{number}
     {}
      {\mkbibparens{\printfield{number}}}%
  \setunit{\addcomma\space}%
  \printfield{eid}%
  \setunit{\addspace}%
  \usebibmacro{issue+date}%
  \setunit{\addcolon\space}%
  \usebibmacro{issue}%
  \newunit}

\usepackage{calc}								% Arithmetische Latex-Befehle





\begin{document}
% Titelseite
\begin{titlepage}
 \centering

\begin{table}[htbp]
 \begin{center}
 \vspace{-0.5cm}
% \renewcommand{\arraystretch}{1.5}
  \begin{tabular}{lcr} 
    \parbox{0.45\textwidth}{\mbox{ }} & \parbox{0.13\textwidth}{\mbox{ }} & \parbox{0.45\textwidth}{\mbox{ }} \\
    \hspace*{-2.0cm}
    \includegraphics[width=1\textwidth]{./Bilder/TUBraunschweig_4C.pdf} &		 					\includegraphics*[width=1\textwidth]{./Bilder/iff_logo_v007.pdf} 
       \\ %empty
  \end{tabular}
 \end{center}
\end{table}


 \vspace*{2.0cm}

 \textbf{\large Protokoll}


 \vspace*{1.5cm}
 
 \textbf{\LARGE Überschrift} \\[0.5ex]


 \vspace*{1.5cm}
 
 \textbf{\large Nico Hempen} \\[0.5ex]
 \textbf{Matrikelnummer 4753519}
 
 \textbf{\large Finn Matz} \\[0.5ex]
 \textbf{Matrikelnummer 4810384}
 
 \textbf{\large -} \\[0.5ex]
 \textbf{Matrikelnummer -------}
 
 \textbf{\large -} \\[0.5ex]
 \textbf{Matrikelnummer -------}
 
 \textbf{\large -} \\[0.5ex]
 \textbf{Matrikelnummer -------}
 

 

 



 \vspace*{2.5cm}

 \begin{table}[htbp]
  \begin{center}
%  \renewcommand{\arraystretch}{1.5}
   \begin{tabular}{rl} 
     \parbox{0.33\textwidth}{\mbox{ }} & \parbox{0.66\textwidth}{\mbox{ }} \\
     Ausgegeben: & Institut f\" ur Flugf\" uhrung \\
                 & Institutsleiter: Prof. Dr. P. Hecker \\
                 & Technische Universit\" at Braunschweig \\
                 &  \\
       Betreuer: & -\\
 %(Erstellt bei:) & (Externe Firma, Stadt) \\
                 %&  \\
 Ver"offentlichung: & Datum \\
   \end{tabular}
  \end{center}
 \end{table}



\end{titlepage}


%only blank page
\newpage
\thispagestyle{empty}
\mbox{}


\pagenumbering{roman}\setcounter{page}{1} 


%% Eidesstattliche Erklaerung (fuer DA, MA und BA)
\chapter*{Eidesstattliche Erkl"arung}\label{s:eid_erkl}


Hiermit erkl"are wir, -, -, -, - und - des Eides statt, das vorliegende Protokoll selbstst"andig und ohne
fremde Hilfe verfasst und keine anderen als die angegebenen Hilfsmittel verwendet zu
haben.

\vspace*{3cm}
Braunschweig, Datum

%\include{empty_page}


% Uebersicht
%\include{abstract}
%\include{empty_page}


% Inhaltsverzeichnis
\fancyhead[RE]{Inhaltsverzeichnis}
\fancyhead[LO]{Inhaltsverzeichnis}
{ 
   \renewcommand{\baselinestretch}{0.85}    %% Zeilenabstand / TOC
   \small\normalsize                        %% neuen Zeilenabstand aktivieren
   \tableofcontents
}
% ATTENTION: comment out if contents sides are even number
%\include{empty_page}


% Nomenklatur
\newpage
\fancyhead[RE]{Nomenklatur}
\fancyhead[LO]{Nomenklatur}
\chapter*{Nomenklatur}
\addcontentsline{toc}{chapter}{Nomenklatur}


\subsection*{Lateinische Bezeichnungen}
\begin{tabbing}
\hspace*{2cm}\=\kill


\end{tabbing}



\subsection*{Griechische Bezeichnungen}
\begin{tabbing}
\hspace*{2cm}\=\kill
$\beta$ \> Winkel in Umfangsrichtung \\[0.2ex]
\end{tabbing}



\subsection*{Indizes}
\begin{tabbing}
\hspace*{2cm}\=\kill
$ax$ \> in axiale Richtung \\[0.2ex]
\end{tabbing}



\subsection*{Abk\"urzungen}
\begin{tabbing}
\hspace*{2cm}\=\kill
$CFD$ \> \underline{C}omputational \underline{F}luid \underline{D}ynamics \\[0.2ex]
\end{tabbing}



% alle Kapitel
\pagenumbering{arabic}\setcounter{page}{1}
\fancyhead[RE]{\nouppercase{\leftmark}}
\fancyhead[LO]{\rightmark}

\chapter{Einleitung (VR)}
\label{c:Einleitung}

Der Versuch Flugmechanik ist Teil des Kompetenzfeldlabors Luft- und Raumfahrttechnik des Studienfachs Maschinenbau und wird vom Institut für Flugführung an der Technischen Universität Braunschweig durchgeführt. Das Labor gliedert sich in Vorbesprechung und anschließendem Flugversuch, einen Laborbericht und ein abschließendes Kolloquium. \\
Im Flugversuch werden stationäre Sinkflüge bei verschiedenen Fluggeschwindigkeiten durchgeführt und die entsprechenden Sinkraten sowie der Treibstoffverbrauch aufgezeichnet. Aus den aufgenommenen Daten lassen sich bei Kenntnis der Atmosphärenbedingungen sowie der Flugzeugmasse und dem Bahnneigungswinkel der Auftriebs- und Widerstandsbeiwert ermitteln. \\
Ziel des Flugversuchs ist es, die aerodynamischen Größen Auftrieb und Widerstand und ihre Beiwerte ohne direkte Kraftmessung zu ermitteln. Über die eigenen aufgezeichneten Werte hinausgehend sollen dieselben Größen auch für die Messreihen der DO 28 berechnet werden. Zusätzlich sollen die Zusammenhänge zwischen Anstellwinkel und Trimmwinkel, Auftriebsbeiwert und Anstellwinkel, sowie Widerstand in Abhängigkeit von der Fluggeschwindigkeit betrachtet werden. Der Umgang mit fehlerbehafteten Messdaten und die grafische Auswertung von Daten stellt ein weiteres Ziel des Labors dar. \\
Nach der Erklärung der theoretischen Grundlagen und der Versuchsdurchführung wird eine Massenabschätzung für den Flugverlauf vorgenommen. Darauf folgend werden die Messdaten ausgewertet und grafisch dargestellt. Abschließend folgt die Interpretation der Ergebnisse durch jeden Teilnehmer der Gruppe und eine Diskussion des Gesamtversuchs, sowie möglicher Fehlerquellen. 






\begin{table}[h]
	\centering
	\begin{tabular}{lr}
		
		Name & \hspace{0.5cm} Initialen\\
		\hline
		Nico Hempen & NH\\
		Tim Gotzel & TG\\
		Finn Matz & FM\\
		Alexander Göhmann & AG \\
		Viktor Rein & VR\\
		\hline
		
	\end{tabular}
	\caption{Initialen der beteiligten Personen}
	\label{tab:initialien}
\end{table}

\chapter{Theoretische Grundlagen (NH)(FM)}
\label{c:TheoGrund}

\section{Luftdichte $\rho$}

Zur Bestimmung der real vorherrschenden Luftdichte in der gegebenen Höhe, wird unter der Annahme, dass Luft ein ideales Gas ist, diese Luftdichte mit der Idealen Gasgleichung definiert:

\begin{equation}
\rho_{real}=\frac{m}{V}=\frac{p}{R_{Luft} \cdot T}
\end{equation}

Dabei kann $R_{Luft}=\SI{287,058 }{\ \joule\kilogram^{-1}\kelvin^{-1}}$ gesetzt werden und der Luftdruck $p$ in definierter Höhe über die Temperatur $T$ mittels der Isentropenbeziehung berechnet werden:

\begin{equation}
p=\left(\frac{T}{T_0}\right)^{\frac{\kappa}{\kappa-1}} \cdot p_0
\end{equation}

Dabei gilt für die Standardbedingungen $T_0=\SI{288,15}{\ \kelvin}$ und $p_0=\SI{101300}{\ \pascal}$.

\section{Wahre Fluggeschwindigkeit $V_{TAS}$}

In den von uns aufgezeichneten Daten der DO-128, sowie in den bereitgestellten Daten der DO-28, liegt die Information der Fluggeschwindigkeit lediglich als \textit{indicated airspeed} vor. Zur Bestimmung der nachfolgenden Beiwerte und Zusammenhänge zwischen den Kenngrößen ist jedoch die so genannte \textit{true airspeed} von Bedeutung. Zur Bestimmung von $V_{TAS}$ aus $V_{IAS}$ wird folgende Formel verwendet\cite{Kurzskript}:

\begin{equation}
V_{TAS}=\sqrt{\frac{\rho_0}{\rho_{real}} \cdot V_{IAS}^2}
\end{equation}

Dabei kann $\rho_0$ als Luftdichte zu $\SI{1,225}{\ \kilogram\meter^{-3}}$ gesetzt werden. Die Fluggeschwindigkeit $V_{IAS}$ muss bei dem Flugversuch mit der DO-128 allerdings noch von $kn$ in $\frac{m}{s}$ umgerechnet werden:

\begin{equation}
V\left(\frac{m}{s}\right)=0,51444 \cdot V\left(kn\right)
\label{eq:V_Umrechnung}
\end{equation}


\section{Auftriebsbeiwert $C_A$}
\label{sec:CA}

Der Auftriebsbeiwert $C_A$ kann per Definition mittels folgender Gleichung bestimmt werden\cite{Skript}:

\begin{equation}
C_A=\frac{A}{\frac{\rho_{real}}{2} \cdot S \cdot V_{TAS}^2}
\label{eq:CA}
\end{equation}

Darin kann der Auftrieb $A$ über die Gewichtskraft $G$ nach folgender Gleichung aufgestellt werden:

\begin{equation}
A=\cos(\gamma) \cdot G
\label{eq:A}
\end{equation}

Da der Bahnwinkel $\gamma$ lediglich in der Messreihe für die DO-28 gegeben ist, muss dieser Wert für die Messreihe der DO-128 über die Sinkgeschwindigkeit $w_{g_{real}}$ und der Fluggeschwindigkeit $V_{TAS}$ bestimmt werden:

\begin{equation}
\gamma=\arcsin\left(\frac{w_{g_{real}}}{V_{TAS}}\right)
\label{eq:gamma}
\end{equation} 

Dabei wird die Sinkgeschwindigkeit $w_{g_{real}}$ bestimmt durch\cite{Kurzskript}:

\begin{equation}
w_{g_{real}}=\frac{\Delta H_{INA}}{\Delta t} \cdot \frac{T_{real}}{T_{INA}}
\end{equation}

Worin $T_{INA}$ für die jeweiligen Höhen aus Tabellen bestimmt werden können und die übrigen Werte im Versuch aufgezeichnet wurden.

Die Flügelfläche $S$ kann in Gleichung \ref{eq:CA} durch die jeweiligen Daten der beiden Flugzeuge ersetzt werden.

\section{Widerstandsbeiwert $C_W$}

Ähnlich wie die Bestimmung des Auftriebsbeiwertes kann auch der Widerstandsbeiwert $C_W$ bestimmt werden:

\begin{equation}
C_W=\frac{W}{\frac{\rho_{real}}{2} \cdot S \cdot V_{TAS}^2}
\end{equation}

Der einzige Unterschied zu $C_A$ besteht in der Verwendung vom Widerstand $W$ statt des Auftriebs $A$ in dieser Gleichung. Dieser kann über die selbe Beziehung wie in Gleichung \ref{eq:A} bestimmt werden:

\begin{equation}
W=\sin(\gamma) \cdot G
\end{equation}

Dabei kann der Bahnneigungswinkel $\gamma$ ebenfalls mit Gleichung \ref{eq:gamma} berechnet werden.

\section{Minimaler Widerstand $W_{min}$}

Durch auftragen des Auftriebsbeiwertes $C_A$ über den Widerstandsbeiwert $C_W$ lassen sich der zum einen der Nullwiderstandsbeiwert $C_{W0}$ und zum anderen die beste Gleitzahl $C_A^*$ bestimmen. $C_{W0}$ ist der Widerstandsbeiwert beim Nullauftrieb, also bei $C_A=0$. $C_A^*$ erhält man durch anlegen einer Tangente, die durch den Ursprung geht. Der Berührungspunkt dieser Tangente mit der Polaren, ist der Punkt des besten Gleitens. Aus diesen beiden Kennwerten lässt sich der Minimale Widerstand $W_{min}$ bestimmen:

\begin{equation}
W_{min}=\frac{2 \cdot C_{W0} \cdot G}{C_A^*}
\label{eq:W_min}
\end{equation}

\section{Optimale Fluggeschwindigkeit $V_{opt}$}

Mit dem bestimmten $C_A^*$ lässt sich zusätzlich die optimale Fluggeschwindigkeit bestimmen, also die Geschwindigkeit, bei der der Widerstand am geringsten ist

\begin{equation}
V_{opt}=\sqrt{\frac{G}{\frac{\rho}{2} \cdot S \cdot C_A^*}}
\label{eq:V_opt}
\end{equation}




\chapter{Versuchsdurchführung (TG)}
\label{c:VdurchF}
tbd
\chapter{Massenabschätzung (AG)}
\label{c:Mabsch}
Die Masse eines Flugzeugs ändert sich durch den Treibstoffverbrauch kontinuierlich während des Fluges. Für die Berechnung der Flugzeugkennwerte ist es wichtig das genaue Flugzeuggewicht zu kennen. Nachfolgend soll die Masse der beiden Flugzeuge Do 28 und Do 128 zu Beginn und Ende jedes Sinkfluges auf Basis der Aufzeichnungen berechnet werden. \\
\textbf{Do 28}\\
Bei der Do 28 wurde von einem konstanten Spritverbrauch für den gesamten Flug ausgegangen. In der Realität verbraucht das Flugzeug bei den Steigflügen mehr Kraftstoff als bei den Sinkflügen. Es war bekannt das die Versuche bei einem Füllstand von 70\% starteten und bei 50\% Füllstand endeten.\\
Maximaler Tankinhalt der Do 28: 822 l\\
Bei einer Kraftstoffdichte von 0,72 kg/l entspricht das 822 l * 0,72 kg/l = 591,84 kg\\
Beginn bei 70\% Tankinhalt: 591,84 kg * 0,7 = 414,29 kg = m\textsubscript{b}\\
Ende bei 50\% Tankinhalt: 591,84 kg * 0,5 = 295,92 kg = m\textsubscript{e}\\
Die Kraftstoffmasse wurde dann mit einer linearen Interpolation \\
$m=m\textsubscript{b}+(\frac{m\textsubscript{e}-m\textsubscript{b}}{t\textsubscript{e}-t\textsubscript{b}})*(t-t\textsubscript{b})$\\
berechnet. (t\textsubscript{b} = 0 s; t\textsubscript{e} = 1860 s)\\
Für die Gesamtmasse wurden die Rüstmasse des Flugzeugs von 2936 kg, die Masse der Besatzung mit 346 kg und die berechnete Kraftstoffmasse addiert. 

\begin{table}[h]
\begin{tabular}{|l|l|l|l|l|l|l|l|l|}
\hline
Do 28                 & \multicolumn{2}{l|}{1. Sinkflug} & \multicolumn{2}{l|}{2. Sinkflug} & \multicolumn{2}{l|}{3. Sinkflug} & \multicolumn{2}{l|}{4. Sinkflug} \\ \hline
                      & Beginn          & Ende           & Beginn          & Ende           & Beginn          & Ende           & Beginn          & Ende           \\ \hline
Zeit t in s           & 60              & 300            & 600             & 920            & 1210            & 1500           & 1760            & 1860           \\ \hline
Kraftstoffmasse in kg & 410,47          & 395,20         & 376,10          & 355,74         & 337,29          & 318,83         & 302,28          & 295,92         \\ \hline
Gesamtmasse in kg     & 3692,47         & 3677,20        & 3658,10         & 3637,74        & 3619,29         & 3600,83        & 3584,28         & 3577,92        \\ \hline
\end{tabular}
\end{table}

\textbf{Do 128}\\
Bei der Do 128 wurde vor und nach jedem Sinkflug der bis zu diesem Zeitpunkt verbrauchte Kraftstoff erfasst. Die Werte wurden in lbs gemessen und nachträglich in kg ungerechnet. [1 lbs/2,20462 = 1 kg]\\
Beim ersten Start befanden sich 617,34 kg Kerosin im Flugzeug, für die Kraftstoffmasse wurde der bis zu einem Zeitpunkt verbrauchte Kraftstoff von der Startmenge subtrahiert. \\
Bei dem Flugversuch wurde vom Piloten eine Rüstmasse von 1388 kg angegeben. Dieser Wert ist zu niedrig. Im Skript ist ein Wert von 3080 kg angegeben, dieser kann abhängig von den verbauten Messinstrumenten abweichen, aber nicht in einem so großen Maß. In Rücksprache mit einer anderen Gruppe konnten wir feststellen, dass vermutlich ein Zahlendreher vorliegt. Dessen Rüstmasse lag bei 3188 kg, diese ist deutlich plausibler und wird bei den Berechnungen verwendet.
Für die Gesamtmasse werden die Rüstmasse und die Masse der Besatzung von 461 kg mit der berechneten Kraftstoffmasse addiert. 
 \\

\begin{table}[h]
\begin{tabular}{|l|l|l|l|l|l|l|l|l|}
\hline
\begin{tabular}[c]{@{}l@{}}Do 128\\   m\textsubscript{r} = 1388 kg\end{tabular} & \multicolumn{2}{l|}{1. Sinkflug} & \multicolumn{2}{l|}{2. Sinkflug} & \multicolumn{2}{l|}{3. Sinkflug} & \multicolumn{2}{l|}{4. Sinkflug} \\ \hline
                                                                & Beginn          & Ende           & Beginn          & Ende           & Beginn          & Ende           & Beginn          & Ende           \\ \hline
Verbraucht lbs                                                  & 237             & 244            & 262             & 267            & 284             & 288            & 304             & 306            \\ \hline
Verbraucht kg                                                   & 107,50          & 110,68         & 118,84          & 121,11         & 128,82          & 130,63         & 137,89          & 138,80         \\ \hline
Kraftstoff total                                                & 509,84          & 506,66         & 498,50          & 496,23         & 488,52          & 486,71         & 479,45          & 478,54         \\ \hline
Gesamt kg                                                       & 2358,84         & 2355,66        & 2347,50         & 2345,23        & 2337,52         & 2335,71        & 2328,45         & 2327,54        \\ \hline
\end{tabular}
\end{table}

\begin{table}[h]
\begin{tabular}{|l|l|l|l|l|l|l|l|l|}
\hline
\begin{tabular}[c]{@{}l@{}}Do 128\\   m\textsubscript{r} = 3188 kg\end{tabular} & \multicolumn{2}{l|}{1. Sinkflug} & \multicolumn{2}{l|}{2. Sinkflug} & \multicolumn{2}{l|}{3. Sinkflug} & \multicolumn{2}{l|}{4. Sinkflug} \\ \hline
                                                                & Beginn          & Ende           & Beginn          & Ende           & Beginn          & Ende           & Beginn          & Ende           \\ \hline
Verbraucht lbs                                                  & 237             & 244            & 262             & 267            & 284             & 288            & 304             & 306            \\ \hline
Verbraucht kg                                                   & 107,50          & 110,68         & 118,84          & 121,11         & 128,82          & 130,63         & 137,89          & 138,80         \\ \hline
Kraftstoff total                                                & 509,84          & 506,66         & 498,50          & 496,23         & 488,52          & 486,71         & 479,45          & 478,54         \\ \hline
Gesamt kg                                                       & 4158,84         & 4155,66        & 4147,50         & 4145,23        & 4137,52         & 4135,71        & 4128,45         & 4127,54        \\ \hline
\end{tabular}
\end{table}

\chapter{Auswertung und Umrechung der Messdaten (AG)}
\label{c:Auswertung}

Messwerte vom Flugversuch mit der Do 128: 

\begin{table}[h]
	\centering
	\begin{tabular}{| l | l | l | l | l | }
\hline
	Do 128 & 1. Sinkflug & 2. Sinkflug & 3. Sinkflug & 4. Sinkflug \\ \hline
	$H_A\ [ft]$ & 4000 & 4000 & 4000 & 4000 \\ \hline
	$H_E\ [ft]$  & 3000 & 3000 & 3000 & 3000 \\ \hline
	$T_A\ [^\circ\text{C}]$  & 12 & 12 & 12 & 12 \\ \hline
	$T_E\ [^\circ\text{C}]$  & 13 & 13 & 13 & 13 \\ \hline
	$V_{IAS}\ [kn]$ & 80 & 100 & 120 & 140 \\ \hline
	$m_a\ [lbs]$ & 237 & 262 & 284 & 304 \\ \hline
	$m_e\ [lbs]$ & 244 & 267 & 288 & 306 \\ \hline
	$\Delta t\ [s]$ & 95 & 67 & 48 & 31 \\ \hline
\end{tabular}
	\caption{Versuchsdaten}
	\label{tab:VersuchDaten2}
\end{table}
Die Werte in der Tabelle werden nun so umgerechnet, dass sie für die folgenden Berechnungen verwendet werden können.\\
Umrechnen von Masse und Temperatur in SI-Einheiten:\\

%\ref{{eq:V_Umrechnung}\\

$T_{[^\circ\text{C}]}+273,15=T_{[K]}$\\
$m_{[lbs]}*0,4536=m_{[kg]}$\\

Umrechnen von $H_{INA}$ in $H_{real}$:\\
Berechnen des Luftdrucks:\\
$p_A=(\frac{T_A}{T_B})^ {\frac{k}{k-1}}*p_0$\\
Mit $T_{Boden}=292,15\ K;\ k=1,4;\ p_0=100400\ Pa$.\\
Berechnen der Luftdichte:\\
$\rho_{real}=\frac{p_A}{R*T_A}$\\
Mit $R=287,058\ Jkg^{-1}K^{-1}$.\\
Berechnen der tatsächlichen Höhe:\\
$H_{real}=H_{INA}*\left(\frac{\rho _{INA}}{\rho _{real}}\right) $\\
Mit $\rho_{INA}=1,225\ kgm^{-3}$ und den gemessenen Höhen in Metern, umgerechnet mit
$H_{[ft]}*0,3048=H_{[m]} $.\\
%mit $\rho _{INA}=1,225\ kgm^{-3}$ und $\rho _{real 4000}=1,\ kgm^{-3}$ und  $\rho _{real 3000}=1,225\ kgm^{-3}$

Die Umrechnung von $V_{IAS_{[kn]}}$ in $V_{TAS_{[ms^{-1}]}}$ wird in den theoretischen Grundlagen \ref{c:TheoGrund} erklärt.
%$V_{[kn]}*0,5144=V_{[ms^{-1}]}$\\
%$V_{TAS_A} = \sqrt{V_{IAS}^2*\frac{\rho_0}{\rho_{real_A}}}$\\
%$V_{TAS_E} = \sqrt{V_{IAS}^2*\frac{\rho_0}{\rho_{real_E}}}$\\
%$V_{TAS} =(V_{TAS_A}+V_{TAS_E})/2$\\
%$\rho_{real}=\frac{p}{R_{Luft} \cdot T}$\\
%mit $\rho_0=1,225\ kgm^{-3}$\\

Umgerechnete Messwerte:

\begin{table}[h]
	\centering
	\begin{tabular}{| l | l | l | l | l | }
\hline
	Do 128 [SI] & 1. Sinkflug & 2. Sinkflug & 3. Sinkflug & 4. Sinkflug \\ \hline
	$H_{real_A}\ [m]$ & 1325,2 & 1325,2 & 1325,2 & 1325,2 \\ \hline
	$H_{real_E}\ [m]$  & 985,2 & 985,2 & 985,2 & 985,2 \\ \hline
	$T_A\ [K]$  & 285,15 & 285,15 & 285,15 & 285,15 \\ \hline
	$T_E\ [K]$  &  286,15 & 286,15 & 286,15 & 286,15 \\ \hline
	$V_{TAS}\ [ms^{-1}]$ & 42,82 & 53,52 & 64,23 & 74,93 \\ \hline
	$m_a\ [kg]$ & 107,50 & 118,84 & 128,82 & 137,89 \\ \hline
	$m_e\ [kg]$ & 110,68 & 121,11 & 130,63 & 138,80 \\ \hline
	$\Delta t\ [s]$ & 95 & 67 & 48 & 31 \\ \hline
\end{tabular}
	\caption{Versuchsdaten in SI-Einheiten}
	\label{tab:VersuchDaten3}
\end{table}

Daten aus der Messreihe der Do 28:\\
Der Staudruck wurde mit $q_{[mBar]}*100=q_{[Pa]}$ umgerechnet.\\

\begin{table}[h]
\begin{tabular}{|l|l|l|l|l|l|l|l|l|}
\hline
Do 28                     & \multicolumn{2}{l|}{1. Sinkflug} & \multicolumn{3}{l|}{2. Sinkflug} & \multicolumn{2}{l|}{3. Sinkflug} & 4. Sinkflug \\ \hline
Nummer & 1               & 2              & 3         & 4         & 5        & 6               & 7              & 8           \\ \hline
$H_A\ [m]$              & 1250            & 875            & 1900      & 1375      & 1000     & 1525            & 700            & 1250        \\ \hline
$H_E\ [m]$              & 1000            & 775            & 1500      & 1150      & 750      & 875             & 500            & 850         \\ \hline
$q\ [pa]$ (gemittelt)    & 19500           & 16900          & 13300     & 10500     & 7600     & 24000           & 7000           & 28000       \\ \hline
$\alpha\ [^\circ]$  (gemittelt) & 4,0             & 5,0            & 7,8       & 9,8       & 15       & 3               & 17,9           & 2,75        \\ \hline
$\eta\ [^\circ]$ (gemittelt)   & -0,6            & -0,7           & -2,5      & -3,1      & -8,1     & 0               & -10,5         & 0,25        \\ \hline
$t\ [s]$                 & 53              & 83             & 63        & 56        & 80       & 101             & 55             & 30          \\ \hline
\end{tabular}
\caption{Messreihe Do 28}
	\label{tab:VersuchDaten4}
\end{table}

\chapter{Darstellung der Ergebnisse}
\label{c:Ergebnisse}

\section{Daten zum Flugversuch der DO-128}

\subsection{Auftriebsbeiwert $C_A$ über Widerstandsbeiwert $C_{W}$}

\begin{figure}[H]
	\centering	\includegraphics[width=0.7\textwidth]{./Bilder/CA_CW_DO128_NEU.jpg}
	\caption{$C_{A}$ über $C_{W}$ der DO-128}
	\label{fig:CA_CW_DO128}
\end{figure}

\subsection{Widerstand $W$ über Fluggeschwindigkeit $V$}
\label{ss:W_V}

\begin{figure}[H]
	\centering	\includegraphics[width=0.7\textwidth]{./Bilder/W_V_DO128_NEU.jpg}
	\caption{$W$ über $V$ der DO-128}
	\label{fig:W_V_DO128}
\end{figure}

Die optimale Fluggeschwindigkeit $V_{opt}$ wurde, wie auch der minimale Widerstand $W_{min}$ (Im Graph nicht zu sehen), mittels der beiden Gleichungen \ref{eq:V_opt} und \ref{eq:W_min} definiert. Dazu wurde die Masse bzw. die Gewichtskraft $G$ des Flugzeugs über die vier Flugabschnitte zu \SI{4142}{\ \kilogram} bzw. \SI{40633}{\ \newton} gemittelt.

\section{Daten zum Flugversuch der DO-28}

\subsection{Anstellwinkel $\alpha$ über Bahnneigungswinkel $\eta$}

\begin{figure}[H]
	\centering	\includegraphics[width=0.7\textwidth]{./Bilder/alpha_eta_plot.jpg}
	\caption{$\alpha$ über $\eta$ der DO-28}
	\label{fig:alpha_eta_DO28}
\end{figure}

\subsection{Auftriebsbeiwert $C_{A}$ über Anstellwinkel $\alpha$}

\begin{figure}[H]
	\centering	\includegraphics[width=0.7\textwidth]{./Bilder/CA_alpha_plot.jpg}
	\caption{$C_{A}$ über $\alpha$ der DO-28}
	\label{fig:CA_alpha_DO28}
\end{figure}

\subsection{Auftriebsbeiwert $C_{A}$ über Widerstandsbeiwert $C_{W}$}

\begin{figure}[H]
	\centering	\includegraphics[width=0.7\textwidth]{./Bilder/CA_CW_DO28_NEU.jpg}
	\caption{$C_{A}$ über $C_{W}$ der DO-28}
	\label{fig:CA_CW_DO28}
\end{figure}

\subsection{Widerstand $W$ über Fluggeschwindigkeit $V$}

\begin{figure}[H]
	\centering	\includegraphics[width=0.7\textwidth]{./Bilder/W_V_DO28_NEU.jpg}
	\caption{$W$ über $V$ der DO-28}
	\label{fig:W_V_DO28}
\end{figure}

Äquivalent zu Abschnitt \ref{ss:W_V} wurde hier $V_{opt}$ und $W_{min}$ mittels der Gleichungen \ref{eq:V_opt} und \ref{eq:W_min} bestimmt. Dabei wurde $G$ aus den 8 Flugabschnitten gemittelt zu \SI{35728}{\ \newton}.

\subsection{Fluggeschwindigkeit $V$ und Staudruck $q$ über Anstellwinkel $\alpha$}

\begin{figure}[H]
	\centering	\includegraphics[width=0.7\textwidth]{./Bilder/V_q_alpha.jpg}
	\caption{$V$ und $q$ über $\alpha$ der DO-28}
	\label{fig:V_q_alpha_DO28}
\end{figure}
\chapter{Interpretation der Ergebnisse (NH)}

\section{Höhenruder Trimmkurve}
tbd

\section{Auftriebsbeiwert über Anstellwinkel}
tbd

\section{Lilienthal-Polare}
tbd

\section{Widerstand über Fluggeschwindigkeit}
tbd

\section{Staudruck über Anstellwinkel}
tbd

\section{Fluggeschwindigkeit über Anstellwinkel}
tbd
\chapter{Interpretation der Ergebnisse (FM)}

In diesem Kapitel sollen die gemessen Werte der beiden Flüge, die im vorherigen Abschnitt aufgetragen wurden, zu interpretieren und die Zusammenhänge zu erklären. 

\section{Höhenruder Trimmkurve}
Die Werte für den Anstellwinkel $\alpha$ und den Trimmwinkel $\eta$ konnten direkt aus den jeweiligen Bereichen der Flugschriebe der DO 28 abgelesen werden. Dazu wurde der Verlauf der Kurve bestmöglich gemittelt. Das Auftragen der erhaltenen Werte lässt einen linearen Zusammenhang vermuten. Die Regressionsgerade mit der Formel $\alpha = -1,4043 \cdot \eta + 3,7239$ bestätigt diesen Verdacht. Es gibt nur sehr geringe Abweichung der Messpunkte von der Geraden. Ein stärkerer Ruderausschlag resultiert also in einer proportionalen Vergrößerung des Anstellwinkels um den Faktor $- 1,4043$.
Demnach wird ein Anstellwinkel von $\alpha = 0^{\circ}$ bei einem Trimmwinkel von $\eta = 2,7^{\circ}$ erreicht. Ein Trimmwinkel von $\eta = 0^{\circ}$ resultiert in einem Anstellwinkel von $\alpha = 3,7239^{\circ}$.

\section{Auftriebsbeiwert über Anstellwinkel}
Der Wert des Anstellwinkels $\alpha$ kann aus den entsprechenden Bereichen des Flugschriebs der DO 28 gemittelt abgelesen werden. Der Auftriebsbeiwert $C_A$ wird durch das in Kapitel \ref{sec:CA} beschriebene Vorgehen bestimmt. 
Das Auftragen der erhaltenen Werte lässt ein linearen Zusammenhang zwischen Anstellwinkel und Auftriebsbeiwert vermuten. Dieses Verhalten entspricht der Theorie. Erst in Bereichen nahe des Strömungsabrisses knickt die gerade nach unten ab \citep{Skript}. In unseren Messwerten ist dieses Verhalten nicht zu erkennen, da der Flugversuch nicht an bis an diese kritischen Bereiche geführt wurde.

Die durchgeführte lineare Regression hat den formelmäßigen Zusammenhang $C_A = 0,0772 \cdot \alpha + 0,2234$. Der Auftriebsbeiwert steigt also bei größer werdendem Anstellwinkel proportional um den Faktor $0,0772$. Dieser Faktor wird auch Auftriebsanstieg oder Derivativ $C_{A \alpha}$ genannt. Per Definition ist er die Ableitung des Auftriebsanstiegs nach dem Anstellwinkel $C_{A \alpha} = \frac{\mathrm{d} C_A}{\mathrm{d} \alpha}$, was bei einer linearen Funktion der Steigung entspricht. 

Bei einem Anstellwinkel von $\alpha = 0^{\circ}$ erhalten wir laut der Regression einen Auftriebsbeiwert von $C_{A0} = 0,2234$. Es entsteht also immer noch Auftrieb durch das Flügelprofil. 
Ein Auftriebsbeiwert von $C_{A0} = 0$ wird erst bei dem Nullauftriebswinkel von $\alpha_0 = -2,894^{\circ}$ erreicht.

\section{Lilienthal-Polare}
Die Lilienthal-Polare wurde sowohl für den Flugversuch der DO 28 als auch für den der DO 128 bestimmt. Sie entsteht durch auftragend des Auftriebsbeiwertes $C_A$ über den Widerstandsbeiwert $C_W$. 
Aus der Theorie erwarten wir einen quadratischen Zusammenhang der Form:

\begin{equation}
C_W = C_{W0} (+ j \cdot C_A) + k \cdot C_{A}^2
\end{equation}

Der Term $k \cdot C_{A}^2$ entspricht dabei dem induzierten Widerstand $C_{W_i}$. Durch gleichsetzten dieses Zusammenhangs mit der Formel für den induzierten Widerstand entsteht eine Möglichkeit den k-Faktor und die Oswald-Zahl $e$ zu bestimmen:

\begin{equation}
C_{W_{i}}=\frac{C_A^2}{\pi \cdot \Lambda}\cdot \textbf{k}=k*C_A^2, \\ \mathrm{mit} \ \Lambda=8,34 
\label{eq:C_W_i}
\end{equation}

\begin{equation}
k = \frac{\textbf{k}}{\pi \cdot \Lambda}
\end{equation}

\textbf{DO 128}

Mit diesen Zusammenhängen lässt sich ein k-Faktor von $0,676$ bestimmen, was einer Oswald-Zahl von $e = 1,5$ entspricht. Diese Kennwerte entsprechen jedoch nicht der Definition, da der k-Faktor stets größer Null ist und die Oswald-Zahl stets kleiner als Null. Die Abweichung der Werte von der Theorie ist wahrscheinlich mit der ungenauen Regression zu erklären.

Anhand des Schnittpunktes der Regression mit der x-Achse, lässt sich der Nullwiderstand $C_{W0} = 0.5$. Dieser Wert liegt durchaus im realistischen Bereich. Weitere Kennwerte lassen sich durch anlegen einer Tangente an die Regression bestimmen, die durch den Ursprung führt. Der Berührungspunkt ist der Punkt des besten Gleitens und es lassen sich sowohl $C_W^* = 0,15$ als auch $C_A^* = 1,39$ ablesen. Aus diesen beiden Werten lassen sich wiederum eine minimale reziproke Gleitzahl von $\epsilon_{min} = 0,1$ und nach Formel \ref{Bahn} der entsprechende Bahnneigungswinkel zu $\gamma=-6,2^{\ \circ}$ bestimmen. Beide Werte sind durchaus im realistischen Rahmen.

\begin{equation}
\tan\left(\gamma\right)=-\frac{C_W}{C_A}
\label{Bahn}
\end{equation}

\textbf{DO 28}

Auffällig bei der Lilienthal-Polaren der DO 28 ist, dass die Messwerte zum Teil deutlich weiter von der Regressionskurve entfernt sind. Dennoch lassen sich die selben Kennwerte bestimmen wie im vorherigen Abschnitt. 
Der k-Faktor, der sich aus der Regressionsformel bestimmen lässt beträgt in diesem Fall $1.6$ und entspricht damit der Definition, dass er größer als 1 ist. Der entsprechende Oswald-Faktor beträgt $e = 0,6$, was ebenfalls der Definition $e<1$ entspricht. 
Der Nullwiderstand, der sich aus dem x-Achsen Schnittpunkt bestimmen lässt ist mit einem Wert von $C_{W0} = 0,068$ etwas zu groß, was eventuell auf die relativ großen Abweichungen zwischen Messwerten und Regression zurückzuführen ist.
Mit der Tangente lassen sich die Kennwerte des besten Gleitens zu $C_A^* = 1,05$ und $C_W^* = 0,092$ bestimmen, was einer minimalen reziproken Gleitzahl von $\epsilon_{min} = 0,09$ entspricht. Daraus lässt sich der Bahnneigungswinkel von $\gamma = -5^{\ \circ}$. Alle diese Werte liegen in einem Bereich, der durchaus üblich für entsprechende Flugzeuge ist.


\section{Widerstand über Fluggeschwindigkeit}
Laut der Theorie setzt sich der Widerstand zum einen aus dem Nullwiderstand $W_0$ und zum anderen aus dem Auftriebswiderstand $W_A$. Beide Komponenten zeigen einen exponentiellen Verlauf über die Fluggeschwindigkeit, jedoch hat die Funktion des Nullwiderstands einen positives Exponenten und steigt, während die Funktion des Auftriebswiderstands einen negativen Exponenten besitzt und somit mit steigender Geschwindigkeit sinkt. Die Überlagerung beider Funktion weißt also einen Tiefpunkt auf, bevor sie ins unendliche ansteigt. Per Definition tritt dieser Tiefpunkt bei der Geschwindigkeit von $V_{opt}$. Da sich $V_{opt}$ beim besten Gleiten einstellt, lässt sich $V_{opt}$ aus den Kennwerten des besten Gleitens nach Formel \ref{eq:V_opt} bestimmen. Für die DO 128 ergibt sich ein Wert von $V_{opt} = 42,21 \frac{m}{s}$. Dieser Wert ist auch im Graphen eingezeichnet, es wird jedoch relativ deutlich, dass dieser Wert nur schlecht zu dem Verlauf passt. Der Tiefpunkt des Graphen scheint erst bei wesentlich geringeren Geschwindigkeiten aufzutreten. Wahrscheinlich stammt dieser Fehler aus der Regression der Lilienthal-Polaren und damit aus Abweichungen bei der Bestimmung von $C_A^*$ und $C_W^*$. Zusätzlich lässt sich der minimale Widerstand nach Gleichung \ref{eq:W_min} berechnen, der theoretisch im Tiefpunkt des Graphen zu finden wäre. Auf diese Weise kann ein Wert von $W_{min} = 2923,28$ N bestimmen. Wenn man den Verlauf der Messwerte im Diagramm der DO 128 \ref{fig:W_V_DO128} betrachtet, scheint dieser relativ gut zu den Werten zu passen.

Die Messwerte der DO 28 sehen allerdings noch schlechter aus. Der erwartete Verlauf lässt sich nur schwer erahnen. Rechnerisch lässt sich $V_{opt} = 44,92 \frac{m}{s}$ zwar bestimmen und in das Diagramm einzeichnen, auch hier ist aber auffällig, dass dieser Wert nur schlecht zu dem Verlauf der Messwerte passt. Nach Gleichung \ref{eq:W_min} lässt sich für die DO 28 ein minimaler Widerstand von $W_{min} = 4678$ N bestimmen. Da wir in der Messreihe allerdings niedrigere Widerstände aufgenommen haben, passt der rechnerisch bestimmte Wert nicht zu erstellten Graphen. Aus diesem Grund haben wir uns dafür entschieden $W_{min}$ nicht in das Diagramm aufzunehmen. Auch in diesem Fall ist die Ursache des Problems sehr wahrscheinlich eine ungünstige quadratische Regression der Lilienthal-Polaren. Der Nullwiderstand $C_{W0}$ ist in der Realität wahrscheinlich deutlich kleiner als der von uns abgelesene Wert, was dazu führt, dass ein zu großer minimaler Widerstand berechnet wurde.

\section{Staudruck und Fluggeschwindigkeit über Anstellwinkel}
In Diagramm \ref{fig:V_q_alpha_DO28} sind sowohl der Staudruck $q$, als auch die wahre Fluggeschwindigkeit $V_{TAS}$ Anstellwinkel $\alpha$ aufgetragen. Es wird deutlich, dass beide Werte einen qualitativ sehr ähnlichen Verlauf zeigen. Das ist in sofern logisch, dass die wahre Fluggeschwindigkeit sich aus dem Staudruck und der Luftdichte berechnen lässt.

\begin{equation}
V_{TAS}= \sqrt{\frac{2 \cdot q}{\rho}}
\end{equation}

Da sich die Luftdichte über die Flugabschnitte nur in geringem Maße ändert, erhalten wir eine nahezu direkte Abhängigkeit der beiden Messreihen, was den ähnlichen verlauf erklärt. 

Mit steigendem Anstellwinkel $\alpha$ sinken sowohl die Fluggeschwindigkeit, als auch der Staudruck. Das ist damit zu erklären, dass ein höherer Anstellwinkel den Auftriebsbeiwert $C_A$ steigert. Je höher der Auftriebsbeiwert, desto geringer ist die Geschwindigkeit, die zum Gleiten benötigt wird. 

\section{Diskussion des Gesamtversuches}
Insgesamt konnten durch den Versuch überwiegend zufriedenstellende Ergebnisse erzielt werden. Der Großteil der Messdaten und ermittelten Kennwerte entsprechen der theoretischen Grundlage. Es gibt allerdings auch ein paar Punkte, in denen die Messdaten sich zum Teil deutlich von den Erwartungen unterscheiden. Ursache für diese Abweichungen ist vor allem das Auswerten der Flugschriebe per Hand. Dabei wurde häufig nur bestmöglich abgelesen und gerundet, wobei stets kleinere Fehler auftreten können. Diese Fehler werden größer, wenn mit ihnen im weiteren Verlauf gerechnet wird und vor allem, wenn Regressionen durch diese unter Umständen ungenauen Messwerte gelegt werden. Vor allem Ausreißer fallen bei einem kleinen Datenumfang stark ins Gewicht. Auf diese Weise ist es möglich, dass unrealistische Ergebnisse entstehen, die nicht mit der Theorie vereinbar sind. In unseren Versuchen stellt das allerdings die Ausnahme dar.
\chapter{Interpretation der Ergebnisse (TG)}

Nachfolgend sollen die Daten, welche jeweils in den Flugversuchen mit den Flugzeugen DO-128 und DO-28 aufgenommen und in den Diagrammen in Abschnitt \ref{c:Ergebnisse} dargestellt worden sind, interpretiert werden. Die Messpunkte in den Graphen der DO-128 resultieren aus den vier Abschnitten, welche erflogen wurden. Für die Messschriebe der Flugversuche mit der DO-28 wurden für alle Versuchsschriebe identisch acht Flugabschnitte definiert (siehe Anhang). Alle Daten wurden anhand dieser ausgewertet. Bestimmend für die Flugabschnitte waren dabei zwei Aspekte: a) Höhe $h$ abnehmend (Sinkflug)  b) Bereich annähernd konstanter Geschwindigkeit $v_{TAS}$. 


\section{Höhenruder Trimmkurve}
Aus den Versuchsschrieben der Kanäle 4 und 5 der DO-28 wurden die Werte der Anstellwinkel $\alpha$ und der Trimmwinkel $\eta$ für die acht Abschnitte jeweils gemittelt abgelesen. In dem Diagramm \ref{fig:alpha_eta_DO28} sind die Werte für $\alpha$ über $\eta$ aufgetragen. Durch die Punkte wurde mittels linearer Regression eine Gerade gelegt. Die angewendete lineare Regressionsformel lautet $\alpha = -1,4043 \cdot \eta + 3,7239$. 

Bei einem Anstellwinkel von $\alpha = 0^{\circ}$ muss ein Trimmwinkel von $\eta = 2,7^{\circ}$ eingestellt werden. Bei einem Trimmwinkel von $\eta = 0^{\circ}$ nimmt der Anstellwinkel den Wert $\alpha = 3,7239^{\circ}$ an. 

An dem Graphen \ref{fig:alpha_eta_DO28} erkennt man sehr gut, dass der Anstellwinkel $\alpha$ und der Trimmwinkel $\eta$ direkt linear voneinander abhängen. Mit stärkeren Ausschlägen des Ruders in negative Richtung steigt ebenfalls der Anstellwinkel proportional um den Faktor $- 1,4043$ an.

Vergleichend mit den Theoriewerten ist dies zu erwartendes Verhalten. Es gibt keine nennenswerten Abweichungen der Messpunkte von der Regressionsgerade. 


\section{Auftriebsbeiwert über Anstellwinkel}
Der Anstellwinkel $\alpha$ wurde in den Versuchsschrieben der DO-28 auf Kanal 4 aufgezeichnet und kann für die acht Abschnitte abgelesen werden.

Der Auftriebsbeiwert $C_A$ kann gemäß Abschnitt \ref{sec:CA} aus der Formel \ref{eq:CA} bestimmt werden. Für die Berechnung gehen die Flügelfläche $S$, die Masse $m$, die Sinkgeschwindigkeit $w_{r_{real}}$, die Fluggeschwindigkeit $v_{TAS}$ und die Luftdichte $\rho_{real}$ ein. Diese Werte können den Messchrieben entnommen werden.

Der Auftriebsbeiwert $C_A$ wurde über den Anstellwinkel $\alpha$ in Diagramm \ref{fig:CA_alpha_DO28} aufgetragen. Dabei stellt sich ein linearer Zusammenhang ein, weswegen erneut eine lineare Regression mit der Geradengleichung $C_A = 0,0772 \cdot \alpha + 0,2234$ angesetzt wurde. Diese Regression genügt für kleine Anstellwinkel der Gleichung 

\begin{equation}
C_A = C_{A \alpha} (\alpha - \alpha_0)
\end{equation}

Es ist erkennbar, dass bei steigendem  Anstellwinkel $\alpha$ auch der Auftriebsbeiwert $C_A$ um den Faktor $C_{A \alpha} = 0,0772$ ansteigt.  
Der Faktor $C_{A \alpha}$ heißt Auftriebsanstieg oder Derivativ und ist per Definition die Ableitung des Auftriebsbeiwerts nach dem Anstellwinkel $C_{A \alpha} = \frac{\mathrm{d} C_A}{\mathrm{d} \alpha}$. 

Bei einem Nullanstellwinkel $\alpha = 0^{\circ}$ nimmt der Auftriebsbeiwert den Wert $C_{A0} = 0,2234$ an. Dass heißt, auch wenn der Flügel nicht angestellt ist, erzeugt er aufgrund seiner Profilgeometrie einen Auftrieb. Damit das Profil keinen Auftrieb mehr erzeugt, muss es um den Nullauftriebswinkel $\alpha_0 = -2,894^{\circ}$ angestellt werden.

Bei größeren Anstellwinkeln würde die Kurve aufgrund von Strömungsabriss am Profil ab $C_{A_{max}}$ nichtlinear stark fallen.

Dieser Graph entspricht der Theorie für kleine Anstellwinkel. Die aufgezeichneten Werte genügen dem erwarteten linearen Verlauf.


\section{Lilienthal-Polare}
Sowohl für die Flugversuche der DO-128 als auch die der DO-28 wurde in den Diagrammen \ref{fig:CA_CW_DO128} und \ref{fig:CA_CW_DO28} der Auftriebsbeiwert $C_A$ über den Widerstandsbeiwert $C_W$ aufgetragen. Weiterhin wurde mittels Polynomansatz zweiten Grades eine Regression durchgeführt und so die quadratische Polare nach folgendem Ansatz bestimmt:

\begin{equation}
C_W = C_{W0} (+ j \cdot C_A) + k \cdot C_{A}^2
\end{equation}

Die Werte der Faktoren j, k und der Wert $C_W0$ für das jeweilige Flugzeug sind in der Legende der Diagramme vermerkt. Diese Polare wird als Lilienthal-Polare bezeichnet. Aus diesem Diagramm kann für verschiedene Bahnwinkel $\gamma = \epsilon = - \frac{C_W}{C_A}$, wobei $\epsilon$ als reziproke Gleitzahl bezeichnet wird, die vorherrschenden Beiwerte ermittelt werden. 

Weiterhin wurden in die beiden Diagramme jeweils eine Tangente vom Ursprung gelegt. Der Winkel dieser Tangente zur x-Achse ist die minimale reziproke Gleitzahl $\epsilon_{min}$. Am Berührungspunkt der Tangente mit der Lilienthal-Polare können die Beiwerte für das beste Gleiten abgelesen werden. Diese sind für die DO-128 $C_{A_{128}}^* = 1,39$ sowie $C_{W_{128}}^* = 0,105$ und für die DO-28 $C_{A_{28}}^* = 1,045$ sowie $C_{W_{28}}^* = 0,0916$. In diesem Flugzustand, nimmt die reziproke Gleitzahl $\epsilon$ den kleinsten Wert an und das Flugzeug gleitet am weitesten. Dies ist jedoch der Zustand, bei dem die Sinkgeschwindigkeit am geringsten ist. Das wird als Fahrt mit minimaler aerodynamischer Verlustleistung bezeichnet und ist bei $C_W = 4 \cdot C_{W0}$ und $C_{A_{wg,min}} = \sqrt{3} \cdot C_A^*$. Das beste Gleiten findet jedoch bereits bei $C_W^* = 2 C_{W0}$ statt. Der Wert $C_{W0}$ wurde mittels Regression für die DO-128 zu $C_{W0_{128}} = 0,05$ und für die DO-28 zu $C_{W0_{28}} = 0,0684$ bestimmt. 


Wie in dem Diagramm \ref{fig:CA_CW_DO128} zu erkennen ist, passt die Regression für die quadratische Polare der DO-128 sehr gut zu den Daten der ermittelten Beiwerte. In Diagramm \ref{fig:CA_CW_DO28} zeigt sich jedoch, dass die Regression der DO-28 nur unzureichend für die Daten der ermittelten Beiwerte passt.


\section{Widerstand über Fluggeschwindigkeit}
Für die Flugzustände wurde 

DO128 Diagramm \ref{fig:W_V_DO128}

DO28 Diagramm \ref{fig:W_V_DO28}

\section{Staudruck über Anstellwinkel}
tbd

DO 28 Diagramm \ref{fig:V_q_alpha_DO28}

\section{Fluggeschwindigkeit über Anstellwinkel}
tbd

DO 28 Diagramm \ref{fig:V_q_alpha_DO28}




\chapter{Interpretation der Ergebnisse (AG)}


\section{Höhenruder Trimmkurve}
Es wurde der Anstellwinkel Alpha über dem Höhenruder Trimmwinkel aufgetragen. Danach wurde eine lineare Regression vorgenommen, die Regressionsgerade hat die Gleichung $\alpha= -1,4043*\eta + 3,7329$. Die aufgetragenen Werte liegen sehr nahe an der Geraden. Ein negatives eta entspricht einem Ruderausschlag nach oben, dann wird am Heck Abtrieb erzeugt und die Flugzeugnase hebt sich. Der Verlauf der Geraden ist dabei realistisch, je größer der Ausschlag am Ruder wird, desto stärker hebt sich die Flugzeugnase, dabei gibt es einen linearen Zusammenhang.
 
\section{Auftriebsbeiwert über Anstellwinkel}
Hier wurde der Auftriebsbeiwert $C_A$ über dem Anstellwinkel $\alpha$ aufgetragen. Danach wurde mit einer linearen Regression eine Regressionsgerade bestimmt, dabei passt die Gerade sehr gut denn alle aufgetragenen Punkte liegen nahe an der Geraden. Die Gleichung der Regressionsgeraden lautet: $C_A=0,0772 \alpha +0,2234$. Die Werte stimmen mit der Theorie überein, je größer der Anstellwinkel wird, desto größer wird auch der Auftrieb des Tragflügels. Bei noch steileren Anstellwinkeln kommt man zu dem Bereich wo die Strömung nicht mehr sauber um den Flügel strömen kann, es kommt zum Strömungsabriss. Die Gerade würde dann schnell nach unten abknicken, dieser Bereich ist im Graphen nicht eingezeichnet. Da es sich um einen profilierten Flügel handelt wird bei einem Anstellwinkel von null Grad immer noch Auftrieb erzeugt, dann liegt der Auftriebsbeiwert bei $C_{A0}$ bei 0,2234. Der Nullauftriebswinkel bei dem der Flügel keinen Auftrieb mehr erzeugt liegt bei $\alpha _0 = -2,894^\circ$. $C_{A0}$ und $\alpha_0$ wurde mit Hilfe der Regressionsgeraden bestimmt.

\section{Lilienthal-Polare}
\textbf {Do 128}\\
Für die Lilienthal-Polare wurde der Auftriebsbeiwert $C_A$ über den Widerstandsbeiwert $C_W$ aufgetragen. Für die eingezeichneten Punkte wurde eine quadratische Regression durchgeführt, die Gleichung der Regressionskurve lautet: $C_W= 0,0503+0,0033\ C_A+0,0258\ {C_A}^2$. Die Punkte liegen sehr nahe an der Kurve und stimmen auch gut mit der Theorie überein. Mit steigendem Auftrieb steigt auch der Widerstand des Flugzeugs immer stärker an, bei wenig Auftrieb steigt der Widerstand nur leicht an. Nach der Regression wurde eine Tangente durch den Ursprung gelegt, beim Berührungspunkt von Tangente und Kurve können nun die Werte für optimales Gleiten $C_A^*$ und $C_W^*$ abgelesen werden, dabei ist $C_A^*=1,39$ und $C_W^*=0,105$. Der Nullwiderstandsbeiwert, den das Flugzeug auch ohne Auftrieb hat, kann am X-Achsenschnittpunkt abgelesen werden und beträgt $C_{W0}=0,05$. Weiterhin kann man in dem Diagramm die minimale reziproke Gleitzahl $\epsilon_{min}=0,076$ bestimmen mit $\epsilon_{min}= C_W^* / C_A^*$. Zusätzlich kann dann noch der Bahnneigungswinkel $\gamma=arctan(-\epsilon)=-4,3^\circ$ berechnet werden. 

\textbf {Do 28}\\
Der Aufbau dieses Graphen für den Lilienthal-Polare ist genauso wie bei der Do 128, auch die Werte werden auf die gleiche Art bestimmt. Der eingetragenen Punkte haben diesmal einen großen Abstand zu der quadratischen Polaren, die Werte sind weit verstreut. Besonders zwei Werte bei $C_W=0,07$ und bei $C_W=0,09$ haben eine besonders große Abweichung von der Kurve. Dadurch ergibt sich eine Kurve bei der bei niedrigen Auftriebsbeiwerten unterhalb von $C_A=0,03$ der Widerstandsbeiwert wieder ansteigt. Dieser Verlauf tritt zum Teil auch bei präziser bestimmten Kurven auf. Die Kurvengleichung lautet: $C_W= 0,0678-0,0441\ C_A+0,0642\ {C_A}^2$. Dann werden $C*_A=1,05$ und $C*_W=0,09$ abgelesen, sowie $C_{W0} =0,07$. Dann ist $\epsilon=0,086$ und $\gamma=4,9^\circ$. 
%Mit den Daten für diesen Graphen wurde noch der Widerstandsanstieg $k=1,6216$ mit $C_W=C_{W0}+k*C_A^2$ berechnet. Mit Hilfe der Flügelstreckung der Do128 von $\Lambda=8,04$

\section{Widerstand über Fluggeschwindigkeit}
\textbf {Do 128}\\
Hier wurde der Widerstand des Flugzeugs Do 128 über die Fluggeschwindigkeit in Metern pro Sekunde aufgetragen. Die einzelnen Messpunkte wurden miteinander verbunden, dabei sieht man das bei steigender Geschwindigkeit der Widerstand immer stärker ansteigt. Der berechnete minimale Widerstand $W_{min}=2923,3\ N$ wurde nicht im Diagramm eingezeichnet, dieser liegt außerhalb der gemessen Werte. Die berechnete optimale Geschwindigkeit $V_{opt}=42,21\  ms^{)-1}$ ist gerade die Geschwindigkeit bei dem der Gesamtwiderstand am geringsten ist. Das passt nicht zusammen da $V_{opt}$ nicht bei  $W_{min}$ liegt. Allerdings beträgt der Widerstand bei $V_{opt}$ ungefähr $3050\ N$ und hat damit nur etwa $125\ N$ Abstand zu $W_{min}$. Bei niedrigeren Geschwindigkeiten würde der Widerstand durch den zunehmenden Auftriebswiderstand wieder ansteigen, dieser Bereich ist im Graphen nicht zu sehen.

\textbf {Do 28}\\
Die Werte sind in diesem Diagramm weit verstreut und wurden deshalb auch nicht verbunden, da sich kein sinnvoller Verlauf ergibt. Die optimale Geschwindigkeit beträgt $V_{opt}=44,92\  ms^{)-1}$. Der berechnete minimale Widerstand liegt bei $W_{min}=4678,4\ N$ und damit sogar oberhalb der meisten Messpunkte. Dieser Fehler lässt sich zum Teil durch den zuvor im $C_A C_W$ Diagramm bestimmten großen Wert von $C_{W0}$ erklären, wobei dort schon einige Messwerte starke Abweichungen hatten. Besonders zwei Punkte bei $V=54\ ms^{-1}$ und bei $V=70\ ms^{-1}$ weichen stark von den restlichen Punkten ab. Ohne diese beiden Punkte kann man sich einen Verlauf vorstellen bei dem bei niedrigen Geschwindigkeiten der Widerstand durch den Auftriebswiderstand noch groß ist, dann kommt der Widerstand zu einem Minimum und steigt dann mit zunehmender Geschwindigkeit immer stärker an. Das würde einem theoretischen Verlauf entsprechen.

\section{Staudruck und Fluggeschwindigkeit über Anstellwinkel}
Staudruck und Fluggeschwindigkeit wurden im gleichen Diagramm über dem Anstellwinkel eingezeichnet und haben ungefähr den gleichen Verlauf. Das ist sinnvoll, da die Fluggeschwindigkeit direkt aus dem Staudruck mit $V=\sqrt{(2*q)/\rho}$ berechnet werden kann. Die beiden Wertereihen folgen ungefähr einem quadratischen Verlauf, auch wenn in diesem Fall auf eine Regression verzichtet wurde. Die Werte sind sinnvoll und passen zur Theorie, je steiler der Anstellwinkel des Flugzeugs wird, desto langsamer wird der Flieger.

\section{Diskussion des Gesamtversuches/Fehlerdiskussion (AG)}
Insgesamt wurden bei dem Flugversuch mit der Do 128 Werte gewonnen die bei der Auswertung dem theoretisch erwarteten Verlauf folgen. Fehler wurden hier bei der Messdatenaufnahme besonders durch die ungenauen Ablesemethoden per Hand gemacht. Eine elektronische Messdatenaufzeichnung hätte genauere Werte geliefert.
Bei den gegebenen Messdaten der Do 28 gibt es größere Abweichungen vor der Theorie, obwohl hier die Daten elektronisch aufgezeichnet wurden. Insbesondere beim $C_A C_W$ Diagramm und beim $W V$ Diagramm streuten die Daten sehr weit. Fehler wurden hier beim Ablesen der gedruckten Messwerte auf dem Papier gemacht. Dieser Fehler könnte durch Verwendung der elektronischen Aufzeichnungen vermieden werden. Auch wurden stark oszillierende Messwerte über einen längeren Zeitraum gemittelt wodurch weiter Fehler entstehen.
Bei den Berechnungen wurden die Masse gemittelt so dass dort am Anfang und Ende der Sinkflüge Abweichungen von der tatsächlichen Masse entstehen. Bei der Do 28 wurde der Kraftstoffverbrauch nur über eine lineare Interpolation bestimmt ohne Momentanverbrauch zu berücksichtigen. Zwischenergebnisse wurden gerundet wenn zu viele Nachkommastellen vorhanden waren.
Trotzdem konnte ein guter Einblick in den Umgang mit realen Messdaten gewonnen werden und auch die Vorgehensweise bei einer Auswertung eines praktischen Versuches konnte geübt werden. Es wurde gezeigt das schon mit wenigen 
\begin {document}

\chapter{Interpretation der Ergebnisse (VR)}

\section{H\"ohenruder Trimmkurve}

\begin{figure}[h]
	\centering
		\includegraphics{C:/Users/vrein_000/Documents/git/Labor IFF/Labor-IFF-Flugversuch/Bilder/alpha_eta_plot.jpg}
	\caption{H\"ohenruder-Trimmkurve}
	\label{fig:alpha_eta_plot}
\end{figure}

Aus den aufgezeichneten Daten l\"asst sich ein linearer Zusammenhang zwischen dem Anstellwinkel alpha und dem 
H\"ohenruderausschlag eta feststellen. Mit steigendem H\"ohenruderausschlag sinkt der Anstellwinkel. 
Der Anstellwinkel \alpha beschreibt den Winkel zwischen dem Fluggeschwindigkeitsvektor und der Flugzeugl\"angsachse. Ein positiver Winkel \alpha bedeutet, dass die Flugzeugl\"angsachse positiv gegen\"uber dem Fluggeschwindigkeitsvektor gedreht ist. \\

\begin{figure} [h]
	\begin{minipage} [b]
		\includegraphics{C:/Users/vrein_000/Documents/git/Labor IFF/Labor-IFF-Flugversuch/Bilder/Anstellwinkel_Definition.jpg}
	\caption{Definition Anstellwinkel}
	\label{alpha_def}
	\end{minipage}

\hspace{0.1\linewidth}
	\begin{minipage} [b]
	\includegraphics{C:/Users/vrein_000/Documents/git/Labor IFF/Labor-IFF-Flugversuch/Bilder/H�henruderausschlag_Definition.jpg}
	\caption{Definition H�henruderausschlag}
	\label{fig:eta_def}
	\end{minipage}
\end{figure}


Der Winkel des H\"ohenruders eta beschreibt die Auslenkung des Ruders gegen\"uber einer Neutrallage. Ein negativer 
H\"ohenruderausschlag bedeutet eine lokale Absenkung des Auftriebs am H\"ohenleitwerk, sodass es zum Absinken des Hecks kommt (Nose-up). In die positive Richtung ausgeschlagen steigt der Auftrieb am H\"ohenleitwerk, sodass sich das Heck hebt (Nosedown). 
Damit decken sich die Messdaten der Do 28 zu Anstellwinkel und H\"ohenruderausschlag mit der Theorie.




\section{Auftriebsbeiwert \"uber Anstellwinkel}

\begin{figure} [t]
	\centering
		\includegraphics{C:/Users/vrein_000/Documents/git/Labor IFF/Labor-IFF-Flugversuch/Bilder/CA_alpha_plot.jpg}
	\caption{Auftriebsbeiwert �ber Anstellwinkel}
	\label{fig:CA_alpha_plot}
	
\end{figure}


Der Auftriebsbeiwert \(C_a\) steigt linear mit zunehmendem Anstellwinkel alpha. \(C_a0\) bezeichnet den Auftriebsbeiwert bei einem Anstellwinkel von null. \(\alpha_0\) beschreibt den Nullauftriebswinkel, an dem der Auftriebsbeiwert null annimmt, das hei{\ss}t kein Auftrieb mehr generiert wird. Der Auftriebsanstieg \(C_a\alpha\) berechnet sich wie folgt: \\
	\[\(C_a\alpha\)=d\(C_a\)\div \(\alpha\)\] \\
	
Bei der Anstr\"omung eines Tragfl\"ugels wird die Luft umgelenkt und es entsteht eine Druckdifferenz zwischen Tragfl\"ugelober- und Oberseite, was als Voraussetzung f\"ur die Entstehung von Auftrieb ist. Der Anstellwinkel des Fl\"ugels ist dabei der gr\"o{\ss}te Einflussfaktor f\"ur den Auftriebsbeiwert. F\"ur kleine Winkel \alpha mit anliegender Str\"omung gilt daher der lineare Zusammenhang
\[\(C_a\)=\(C_a\alpha\)(\alpha-\(alpha_0\))\] \\

Damit stimmen die Messdaten mit der Theorie \"uberein. Jedoch lassen sich keine Aussagen \"uber den maximalen Auftriebsbeiwert treffen, da die aufgenommenen Anstellwinkel im Bereich der anliegenden Str\"omung liegen und der h\"ochste Auftriebsbeiwert erst kurz vor dem Abrei{\ss}en der Str\"omung erreicht wird. 



\section{Lilienthal-Polare}
\begin{figure} [h]
		\includegraphics{C:/Users/vrein_000/Documents/git/Labor IFF/Labor-IFF-Flugversuch/Bilder/CA_CW_DO28_NEU.jpg}
	\caption{Do 28}
	\label{fig:CA_CW_DO28}
\end{figure}

Die f�r \(C_A\) und \(C_W\) errechneten Werte werden f�r die Erstellung der Lilienthalpolare gegeneinnder aufgetragen und mit einer quadratischen Regression angen�hert. Daraus ergibt sich f�r den Nullwiderstand \(C_W0\) ein Wert von 0,0503, f�r den Polynomterm erster Ordnung ein Koeffizient von 0,0033 und f�r den zweiten Grades ein Koeffizient von 0,0258. 
Aus der Lillienthal-Polare l�sst sich die minimale reziproke Gleitzahl ermitteln, die die H�hendifferenz beim Sinken des Flugzeugs bei einer bestimmten Flugstrecke beschreibt. Zur Ermittlung der minimalen reziproken Gleitzahl wird eine Tangente durch den Ursprung an die entstandene Polare gelegt. Am Schnittpunkt der Tangente mit der Polaren k�nnen die zugeh�rigen Beiwerte \(C^{*}_A\)=1,39 und  \(C^{*}_W\)=0,105 abgelesen werden, mit denen die reziproke Steigung der Tangente mit

\[ \(\epsilon_min\)=  \(C^{*}_W\) \div \(C^{*}_A\)=0,105 \div 1,39=0,0755\] \\

berechnet werden kann. Der Gleitwinkel \gamma, der die Drehung des Geschwindigkeitsvektor gegen�ber der geod�tischen Horizontalebene in x-Richtung beschreibt, ergibt sich zu

\[\gamma=\arctan(- (\(C^{*}_W\) \div \(C\{*}_)^A\])) = arctan(-(0,105 \div 1,39)) = -4,32~�\]  \\

Aus den bekannten Werten \(C_W0\) und \(C^{*}_A\) l�sst sich der Widerstandsanstieg k wie folgt berechnen: 

\[k= \(C_W0\) \div  \(C^{*} _A\)^{2}=0,0451 \div 1,39^{2}=0,2334 \] \\

 In der Theorie wird an dieser Stelle ein positiver Wert f�r k erwartet, was bedeutet, dass die aus der Regression entnommenen Werte falsch sind. Da nur vier verschiedene Flugzust�nde aufgenommen werden konnten, ist die Regression au�erhalb des Messwertbereichs ungenau. 





\begin{figure}	
	\includegraphics{C:/Users/vrein_000/Documents/git/Labor IFF/Labor-IFF-Flugversuch/Bilder/CA_CW_DO128_NEU.jpg}
	\caption{Do 128}
	\label{fig:CA_CW_DO128_NEU}
\end{figure}




\section{Widerstand \"uber Fluggeschwindigkeit}
tbd

\section{Staudruck \"uber Anstellwinkel}
tbd

\section{Fluggeschwindigkeit \"uber Anstellwinkel}
tbd




\end {document}
\input{Kapitel/Anhang}

%only blank page
\newpage
\thispagestyle{empty}
\mbox{}


% Literaturverzeichnis
\newpage
\addcontentsline{toc}{chapter}{Literatur}
\printbibliography


\end{document}