\chapter{Versuchsdurchführung (TG)}
\label{c:VdurchF}

In diesem Kapitel wird die Versuchsdurchführung für den Flug mit der DO128 der Technischen Universität Braunschweig beschrieben. 

Der Flug wurde am 20.05.2019 mit 6 Mann Besatzung durchgeführt und dauerte 20 Minuten. Die vorherrschenden Umweltparameter am Boden wurden notiert und sind in Tabelle \ref{tab:VersuchDaten1} festgehalten. 


\begin{table}[h]
	\centering
	\begin{tabular}{| l | l | }
\hline
	Datum  & 20.05.2019 \\ \hline
	Beginn Flug & 9:47 \\ \hline
	Ende Flug & 10:10 \\ \hline
	Besatzung (Masse) & 461 kg \\ \hline
	Höhenmessereinstellung & 1013,25 hPa QNH \\ \hline
	Rüstmasse & 1388 kg \\ \hline
	Kraftstoffmasse am Boden & 1361 lbs \\ \hline
	Wetter & sonnig, leichte Quellwolken \\ \hline
	Temperatur  & \SI{19}{\celsius} \\ \hline
	Wind & 5 kn / 060 \\ \hline
	Druck (Platzhöhe)  & 1004 hPa  \\ \hline
	\end{tabular}
	\caption{Parameter am Versuchstag}
	\label{tab:VersuchDaten1}
\end{table}

Nach dem Einstellen des Höhenmessers auf QNH startete die DO 128 und nahm ihre Zielhöhe knapp über 4000 ft ein. Für die Versuche wurden 4 Sinkflüge bei verschiedenen Fluggeschwindigkeiten (80 kn, 100 kn, 120 kn, 140 kn) absolviert. Dabei beschleunigte der Pilot das Flugzeug auf die Sollgeschwindigkeit. Beim erreichen der 4000 ft Marke stoppte der Copilot die Zeit die benötigt wurde, um 1000 ft zu sinken, mit der Stoppuhr. Dabei wurden am Ende und am Anfang die Temperaturen und die verbrauchte Treibstoffmasse auf der 4000 ft Marke und auf der 3000 ft Marke von dem Copiloten abgelesen. Alle relevanten Daten wurden vom Copiloten per Mikro an die Besatzung weitergegeben, welche diese in die vorbereiteten Versuchsprotokolle notierte. Die Werte sind in Tabelle \ref{tab:VersuchDaten2} zu sehen. Nachdem ein Sinkflug absolviert war, stieg die DO 128 wieder auf knapp über 4000 ft und ein erneuter Sinkflug wurde eingeleitet. 


\begin{table}[H]
	\centering
	\begin{tabular}{| l | l | l | l | l | }
\hline
	Sinkflug & 1 & 2 & 3 & 4 \\ \hline
	$H_A$ [ft] & 4000 & 4000 & 4000 & 4000 \\ \hline
	$H_E$ [ft]  & 3000 & 3000 & 3000 & 3000 \\ \hline
	$T_A$  \SI{}{\celsius} & 12 & 12 & 12 & 12 \\ \hline
	$T_E$  \SI{}{\celsius} & 13 & 13 & 13 & 13 \\ \hline
	$V_{IAS}$ [kn] & 80 & 100 & 120 & 140 \\ \hline
	$m_a$ [lbs] & 237 & 262 & 284 & 304 \\ \hline
	$m_e$ [lbs] & 244 & 267 & 288 & 306 \\ \hline
	$\Delta$ t [s] & 95 & 67 & 48 & 31 \\ \hline
\end{tabular}
	\caption{Versuchsdaten}
	\label{tab:VersuchDaten2}
\end{table}

Bei dem Sinkflug handelte es sich um einen Gleiten bei verschiedenen Fluggeschwindigkeiten. Die Propeller der DO-128 lieferten nur so viel Schub, um  die Verluste zu kompensieren, die sie selber erzeugten.
