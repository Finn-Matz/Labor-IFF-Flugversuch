\chapter{Interpretation der Ergebnisse (FM)}

In diesem Kapitel sollen die gemessen Werte der beiden Flüge, die im vorherigen Abschnitt aufgetragen wurden, zu interpretieren und die Zusammenhänge zu erklären. 

\section{Höhenruder Trimmkurve}
Die Werte für den Anstellwinkel $\alpha$ und den Trimmwinkel $\eta$ konnten direkt aus den jeweiligen Bereichen der Flugschriebe der DO-28 abgelesen werden. Dazu wurde der Verlauf der Kurve bestmöglich gemittelt. Das Auftragen der erhaltenen Werte lässt einen linearen Zusammenhang vermuten. Die Regressionsgerade mit der Formel $\alpha = -1,4043 \cdot \eta + 3,7239$ bestätigt diesen Verdacht. Es gibt nur sehr geringe Abweichung der Messpunkte von der Geraden. Ein stärkerer Ruderausschlag resultiert also in einer proportionalen Vergrößerung des Anstellwinkels um den Faktor $- 1,4043$.
Demnach wird ein Anstellwinkel von $\alpha = 0^{\circ}$ bei einem Trimmwinkel von $\eta = 2,7^{\circ}$ erreicht. Ein Trimmwinkel von $\eta = 0^{\circ}$ resultiert in einem Anstellwinkel von $\alpha = 3,7239^{\circ}$.

\section{Auftriebsbeiwert über Anstellwinkel}
Der Wert des Anstellwinkels $\alpha$ kann aus den entsprechenden Bereichen des Flugschriebs der DO-28 gemittelt abgelesen werden. Der Auftriebsbeiwert $C_A$ wird durch das in Kapitel \ref{sec:CA} beschriebene Vorgehen bestimmt. 
Das Auftragen der erhaltenen Werte lässt ein linearen Zusammenhang zwischen Anstellwinkel und Auftriebsbeiwert vermuten. Dieses Verhalten entspricht der Theorie. Erst in Bereichen nahe des Strömungsabrisses knickt die gerade nach unten ab \citep{Skript}. In unseren Messwerten ist dieses Verhalten nicht zu erkennen, da der Flugversuch nicht an bis an diese kritischen Bereiche geführt wurde.

Die durchgeführte lineare Regression hat den formelmäßigen Zusammenhang $C_A = 0,0772 \cdot \alpha + 0,2234$. Der Auftriebsbeiwert steigt also bei größer werdendem Anstellwinkel proportional um den Faktor $0,0772$. Dieser Faktor wird auch Auftriebsanstieg oder Derivativ $C_{A \alpha}$ genannt. Per Definition ist er die Ableitung des Auftriebsanstiegs nach dem Anstellwinkel $C_{A \alpha} = \frac{\mathrm{d} C_A}{\mathrm{d} \alpha}$, was bei einer linearen Funktion der Steigung entspricht. 

Bei einem Anstellwinkel von $\alpha = 0^{\circ}$ erhalten wir laut der Regression einen Auftriebsbeiwert von $C_{A0} = 0,2234$. Es entsteht also immer noch Auftrieb durch das Flügelprofil. 
Ein Auftriebsbeiwert von $C_{A0} = 0$ wird erst bei dem Nullauftriebswinkel von $\alpha_0 = -2,894^{\circ}$ erreicht.

\section{Lilienthal-Polare}
Die Lilienthal-Polare wurde sowohl für den Flugversuch der DO-28 als auch für den der DO-128 bestimmt. Sie entsteht durch auftragend des Auftriebsbeiwertes $C_A$ über den Widerstandsbeiwert $C_W$. 
Aus der Theorie erwarten wir einen quadratischen Zusammenhang der Form:

\begin{equation}
C_W = C_{W0} (+ j \cdot C_A) + k \cdot C_{A}^2
\end{equation}

Der Therm $k \cdot C_{A}^2$ entspricht dabei dem induzierten Widerstand $C_{W_i}$. Durch gleichsetzten dieses Zusammenhangs mit der Formel für den induzierten Widerstand entsteht eine Möglichkeit den k-Faktor und die Oswald-Zahl $e$:

\begin{equation}
C_{W_{i}}=\frac{C_A^2}{\pi \cdot \Lambda}\cdot \textbf{k}=k*C_A^2, \\ \mathrm{mit} \ \Lambda=8,34 
\label{eq:C_W_i}
\end{equation}

\begin{equation}
k = \frac{\textbf{k}}{\pi \cdot \Lambda}
\end{equation}

\subsection{DO-128}

Mit diesen Zusammenhängen lässt sich ein k-Faktor von $0,676$ bestimmen, was einer Oswald-Zahl von $e = 1,5$ entspricht. Diese Kennwerte entsprechen jedoch nicht der Definition, da der k-Faktor stets größer Null ist und die Oswald-Zahl stets kleiner als Null. Die Abweichung der Werte von der Theorie ist wahrscheinlich mit der ungenauen Regression zu erklären.

Anhand des Schnittpunktes der Regression mit der x-Achse, lässt sich der Nullwiderstand $C_{W0} = 0.5$. Dieser Wert liegt durchaus im realistischen Bereich. Weitere Kennwerte lassen sich durch anlegen einer Tangente an die Regression bestimmen, die durch den Ursprung führt. Der Berührungspunkt ist der Punkt des besten Gleitens und es lassen sich sowohl $C_W^* = 0,15$ als auch $C_A^* = 1,39$ ablesen. Aus diesen beiden Werten lassen sich wiederum eine minimale reziproke Gleitzahl von $\epsilon_{min} = 0,1$ und nach Formel \ref{Bahn} der entsprechende Bahnneigungswinkel zu $\gamma=-6,2^{\ \circ}$ bestimmen. Beide Werte sind durchaus im realistischen Rahmen.

\begin{equation}
\tan\left(\gamma\right)=-\frac{C_W}{C_A}
\label{Bahn}
\end{equation}

\subsection{DO-28}

Auffällig bei der Lilienthal-Polaren der DO-28 ist, dass die Messwerte zum Teil deutlich weiter von der Regressionskurve entfernt sind. Dennoch lassen sich die selben Kennwerte bestimmen wie im vorherigen Abschnitt. 
Der k-Faktor, der sich aus der Regressionsformel bestimmen lässt beträgt in diesem Fall $1.6$ und entspricht damit der Definition, dass er größer als 1 ist. Der entsprechende Oswald-Faktor beträgt $e = 0,6$, was ebenfalls der Definition $e<1$ entspricht. 
Der Nullwiderstand, der sich aus dem x-Achsen Schnittpunkt bestimmen lässt ist mit einem Wert von $C_{W0} = 0,068$ etwas zu groß, was eventuell auf die relativ großen Abweichungen zwischen Messwerten und Regression zurückzuführen ist.
Mit der Tangente lassen sich die Kennwerte des besten Gleitens zu $C_A^* = 1,05$ und $C_W^* = 0,092$ bestimmen, was einer minimalen reziproken Gleitzahl von $\epsilon_{min} = 0,09$ entspricht. Daraus lässt sich der Bahnneigungswinkel von $\gamma = -5^{\ \circ}$. Alle diese Werte liegen in einem Bereich, der durchaus üblich für entsprechende Flugzeuge ist.


\section{Widerstand über Fluggeschwindigkeit}
Laut der Theorie setzt sich der Widerstand zum einen aus dem Nullwiderstand $W_0$ und zum anderen aus dem Auftriebswiderstand $W_A$. Beide Komponenten zeigen einen exponentiellen Verlauf über die Fluggeschwindigkeit, jedoch hat die Funktion des Nullwiderstands einen positives Exponenten und steigt, während die Funktion des Auftriebswiderstands einen negativen Exponenten besitzt und somit mit steigender Geschwindigkeit sinkt. Die Überlagerung beider Funktion weißt also einen Tiefpunkt auf, bevor sie ins unendliche ansteigt. Per Definition tritt dieser Tiefpunkt bei der Geschwindigkeit von $V_{opt}$. Da sich $V_{opt}$ beim besten Gleiten einstellt, lässt sich $V_{opt}$ aus den Kennwerten des besten Gleitens nach Formel \ref{eq:V_opt} bestimmen. Für die DO-128 ergibt sich ein Wert von $V_{opt} = 42,21 \frac{m}{s}$. Dieser Wert ist auch im Graphen eingezeichnet, es wird jedoch relativ deutlich, dass dieser Wert nur schlecht zu dem Verlauf passt. Der Tiefpunkt des Graphen scheint erst bei wesentlich geringeren Geschwindigkeiten aufzutreten. Wahrscheinlich stammt dieser Fehler aus der Regression der Lilienthal-Polaren und damit aus Abweichungen bei der Bestimmung von $C_A^*$ und $C_W^*$.

Noch schlechter sehen allerdings die Messwerte der DO-28 aus. Der erwartete Verlauf lässt sich nur schwer erahnen. Rechnerisch lässt sich $V_{opt} = 44,92 \frac{m}{s}$
bestimmen und in das Diagramm einzeichnen. Auch hier ist auffällig, dass der rechnerisch bestimmte Wert nur schlecht zu dem Verlauf der Messwerte passt.
\section{Staudruck über Anstellwinkel}
tbd

\section{Fluggeschwindigkeit über Anstellwinkel}
tbd