\chapter{Interpretation der Ergebnisse (AG)}


\section{Höhenruder Trimmkurve}
Es wurde der Anstellwinkel Alpha über dem Höhenruder Trimmwinkel aufgetragen. Danach wurde eine lineare Regression vorgenommen, die Regressionsgerade hat die Gleichung $\alpha= -1,4043*\eta + 3,7329$. Die aufgetragenen Werte liegen sehr nahe an der Geraden. Ein negatives eta entspricht einem Ruderausschlag nach oben, dann wird am Heck Abtrieb erzeugt und die Flugzeugnase hebt sich. Der Verlauf der Geraden ist dabei realistisch, je größer der Ausschlag am Ruder wird, desto stärker hebt sich die Flugzeugnase, dabei gibt es einen linearen Zusammenhang.
 
\section{Auftriebsbeiwert über Anstellwinkel}
Hier wurde der Auftriebsbeiwert $C_A$ über dem Anstellwinkel $\alpha$ aufgetragen. Danach wurde mit einer linearen Regression eine Regressionsgerade bestimmt, dabei passt die Gerade sehr gut denn alle aufgetragenen Punkte liegen nahe an der Geraden. Die Gleichung der Regressionsgeraden lautet: $C_A=0,0772 \alpha +0,2234$. Die Werte stimmen mit der Theorie überein, je größer der Anstellwinkel wird, desto größer wird auch der Auftrieb des Tragflügels. Bei noch steileren Anstellwinkeln kommt man zu dem Bereich wo die Strömung nicht mehr sauber um den Flügel strömen kann, es kommt zum Strömungsabriss. Die Gerade würde dann schnell nach unten abknicken, dieser Bereich ist im Graphen nicht eingezeichnet. Da es sich um einen profilierten Flügel handelt wird bei einem Anstellwinkel von null Grad immer noch Auftrieb erzeugt, dann liegt der Auftriebsbeiwert bei $C_{A0}$ bei 0,2234. Der Nullauftriebswinkel bei dem der Flügel keinen Auftrieb mehr erzeugt liegt bei $\alpha _0 = -2,894^\circ$. $C_{A0}$ und $\alpha_0$ wurde mit Hilfe der Regressionsgeraden bestimmt.

\section{Lilienthal-Polare}
\textbf {Do 128}\\
Für die Lilienthal-Polare wurde der Auftriebsbeiwert $C_A$ über den Widerstandsbeiwert $C_W$ aufgetragen. Für die eingezeichneten Punkte wurde eine quadratische Regression durchgeführt, die Gleichung der Regressionskurve lautet: $C_W= 0,0503+0,0033\ C_A+0,0258\ {C_A}^2$. Die Punkte liegen sehr nahe an der Kurve und stimmen auch gut mit der Theorie überein. Mit steigendem Auftrieb steigt auch der Widerstand des Flugzeugs immer stärker an, bei wenig Auftrieb steigt der Widerstand nur leicht an. Nach der Regression wurde eine Tangente durch den Ursprung gelegt, beim Berührungspunkt von Tangente und Kurve können nun die Werte für optimales Gleiten $C_A^*$ und $C_W^*$ abgelesen werden, dabei ist $C_A^*=1,39$ und $C_W^*=0,105$. Der Nullwiderstandsbeiwert, den das Flugzeug auch ohne Auftrieb hat, kann am X-Achsenschnittpunkt abgelesen werden und beträgt $C_{W0}=0,05$. Weiterhin kann man in dem Diagramm die minimale reziproke Gleitzahl $\epsilon_{min}=0,076$ bestimmen mit $\epsilon_{min}= C_W^* / C_A^*$. Zusätzlich kann dann noch der Bahnneigungswinkel $\gamma=arctan(-\epsilon)=-4,3^\circ$ berechnet werden. 

\textbf {Do 28}\\
Der Aufbau dieses Graphen für den Lilienthal-Polare ist genauso wie bei der Do 128, auch die Werte werden auf die gleiche Art bestimmt. Der eingetragenen Punkte haben diesmal einen großen Abstand zu der quadratischen Polaren, die Werte sind weit verstreut. Besonders zwei Werte bei $C_W=0,07$ und bei $C_W=0,09$ haben eine besonders große Abweichung von der Kurve. Dadurch ergibt sich eine Kurve bei der bei niedrigen Auftriebsbeiwerten unterhalb von $C_A=0,03$ der Widerstandsbeiwert wieder ansteigt. Dieser Verlauf tritt zum Teil auch bei präziser bestimmten Kurven auf. Die Kurvengleichung lautet: $C_W= 0,0678-0,0441\ C_A+0,0642\ {C_A}^2$. Dann werden $C*_A=1,05$ und $C*_W=0,09$ abgelesen, sowie $C_{W0} =0,07$. Dann ist $\epsilon=0,086$ und $\gamma=4,9^\circ$. 
%Mit den Daten für diesen Graphen wurde noch der Widerstandsanstieg $k=1,6216$ mit $C_W=C_{W0}+k*C_A^2$ berechnet. Mit Hilfe der Flügelstreckung der Do128 von $\Lambda=8,04$

\section{Widerstand über Fluggeschwindigkeit}
\textbf {Do 128}\\
Hier wurde der Widerstand des Flugzeugs Do 128 über die Fluggeschwindigkeit in Metern pro Sekunde aufgetragen. Die einzelnen Messpunkte wurden miteinander verbunden, dabei sieht man das bei steigender Geschwindigkeit der Widerstand immer stärker ansteigt. Der berechnete minimale Widerstand $W_{min}=2923,3\ N$ wurde nicht im Diagramm eingezeichnet, dieser liegt außerhalb der gemessen Werte. Die berechnete optimale Geschwindigkeit $V_{opt}=42,21\  ms^{)-1}$ ist gerade die Geschwindigkeit bei dem der Gesamtwiderstand am geringsten ist. Das passt nicht zusammen da $V_{opt}$ nicht bei  $W_{min}$ liegt. Allerdings beträgt der Widerstand bei $V_{opt}$ ungefähr $3050\ N$ und hat damit nur etwa $125\ N$ Abstand zu $W_{min}$. Bei niedrigeren Geschwindigkeiten würde der Widerstand durch den zunehmenden Auftriebswiderstand wieder ansteigen, dieser Bereich ist im Graphen nicht zu sehen.

\textbf {Do 28}\\
Die Werte sind in diesem Diagramm weit verstreut und wurden deshalb auch nicht verbunden, da sich kein sinnvoller Verlauf ergibt. Die optimale Geschwindigkeit beträgt $V_{opt}=44,92\  ms^{)-1}$. Der berechnete minimale Widerstand liegt bei $W_{min}=4678,4\ N$ und damit sogar oberhalb der meisten Messpunkte. Dieser Fehler lässt sich zum Teil durch den zuvor im $C_A C_W$ Diagramm bestimmten großen Wert von $C_{W0}$ erklären, wobei dort schon einige Messwerte starke Abweichungen hatten. Besonders zwei Punkte bei $V=54\ ms^{-1}$ und bei $V=70\ ms^{-1}$ weichen stark von den restlichen Punkten ab. Ohne diese beiden Punkte kann man sich einen Verlauf vorstellen bei dem bei niedrigen Geschwindigkeiten der Widerstand durch den Auftriebswiderstand noch groß ist, dann kommt der Widerstand zu einem Minimum und steigt dann mit zunehmender Geschwindigkeit immer stärker an. Das würde einem theoretischen Verlauf entsprechen.

\section{Staudruck und Fluggeschwindigkeit über Anstellwinkel}
Staudruck und Fluggeschwindigkeit wurden im gleichen Diagramm über dem Anstellwinkel eingezeichnet und haben ungefähr den gleichen Verlauf. Das ist sinnvoll, da die Fluggeschwindigkeit direkt aus dem Staudruck mit $V=\sqrt{(2*q)/\rho}$ berechnet werden kann. Die beiden Wertereihen folgen ungefähr einem quadratischen Verlauf, auch wenn in diesem Fall auf eine Regression verzichtet wurde. Die Werte sind sinnvoll und passen zur Theorie, je steiler der Anstellwinkel des Flugzeugs wird, desto langsamer wird der Flieger.

\section{Diskussion des Gesamtversuches/Fehlerdiskussion (AG)}
Insgesamt wurden bei dem Flugversuch mit der Do 128 Werte gewonnen die bei der Auswertung dem theoretisch erwarteten Verlauf folgen. Fehler wurden hier bei der Messdatenaufnahme besonders durch die ungenauen Ablesemethoden per Hand gemacht. Eine elektronische Messdatenaufzeichnung hätte genauere Werte geliefert.
Bei den gegebenen Messdaten der Do 28 gibt es größere Abweichungen vor der Theorie, obwohl hier die Daten elektronisch aufgezeichnet wurden. Insbesondere beim $C_A C_W$ Diagramm und beim $W V$ Diagramm streuten die Daten sehr weit. Fehler wurden hier beim Ablesen der gedruckten Messwerte auf dem Papier gemacht. Dieser Fehler könnte durch Verwendung der elektronischen Aufzeichnungen vermieden werden. Auch wurden stark oszillierende Messwerte über einen längeren Zeitraum gemittelt wodurch weiter Fehler entstehen.
Bei den Berechnungen wurden die Masse gemittelt so dass dort am Anfang und Ende der Sinkflüge Abweichungen von der tatsächlichen Masse entstehen. Bei der Do 28 wurde der Kraftstoffverbrauch nur über eine lineare Interpolation bestimmt ohne Momentanverbrauch zu berücksichtigen. Zwischenergebnisse wurden gerundet wenn zu viele Nachkommastellen vorhanden waren.
Trotzdem konnte ein guter Einblick in den Umgang mit realen Messdaten gewonnen werden und auch die Vorgehensweise bei einer Auswertung eines praktischen Versuches konnte geübt werden.  Es wurde gezeigt das schon mit wenigen Messdaten die relevanten Kennwerte eines Flugzeuges bestimmt werden können.