\chapter{Einleitung (VR)}
\label{c:Einleitung}

Der Versuch Flugmechanik ist Teil des Kompetenzfeldlabors Luft- und Raumfahrttechnik des Studienfachs Maschinenbau und wird vom Institut für Flugführung an der Technischen Universität Braunschweig durchgeführt. Das Labor gliedert sich in Vorbesprechung und anschließendem Flugversuch, einen Laborbericht und ein abschließendes Kolloquium. 
Im Flugversuch werden stationäre Sinkflüge bei verschiedenen Fluggeschwindigkeiten durchgeführt und die entsprechenden Sinkraten sowie der Treibstoffverbrauch aufgezeichnet. Aus den aufgenommenen Daten lassen sich bei Kenntnis der Atmosphärenbedingungen sowie der Flugzeugmasse und dem Bahnneigungswinkel der Auftriebs- und Widerstandsbeiwert ermitteln. //
Ziel des Flugversuchs ist es, die aerodynamischen Größen Auftrieb und Widerstand und ihre Beiwerte ohne direkte Kraftmessung zu ermitteln. über die eigenen aufgezeichneten Werte hinausgehend sollen dieselben Größen auch für die Messreihen der DO 28 berechnet werden. Zusätzlich sollen die Zusammenhänge zwischen Anstellwinkel und Trimmwinkel, Auftriebsbeiwert und Anstellwinkel und Widerstand in Abhängigkeit von der Fluggeschwindigkeit betrachtet werden. Der Umgang mit fehlerbehafteten Messdaten und die grafische Auswertung von Daten stellt ein weiteres Ziel des Labors dar. //
Nach der Erklärung der theoretischen Grundlagen und der Versuchsdurchführung wird eine Massenabschätzung für den Flugverlauf vorgenommen. Darauf folgend werden die Messdaten ausgewertet und grafisch dargestellt. Abschließend folgt die Interpretation der Ergebnisse durch jeden Teilnehmer der Gruppe und eine Diskussion des Gesamtversuchs und möglicher Fehlerquellen. 






\begin{table}[h]
	\centering
	\begin{tabular}{lr}
		
		Name & \hspace{0.5cm} Initialen\\
		\hline
		Nico Hempen & NH\\
		Tim Gotzel & TG\\
		Finn Matz & FM\\
		Alexander Göhmann & AG \\
		Viktor Rein & VR\\
		\hline
		
	\end{tabular}
	\caption{Initialen der beteiligten Personen}
	\label{tab:initialien}
\end{table}
