\chapter{Interpretation der Ergebnisse (NH)}

\section{Höhenruder Trimmkurve}

Im Flugversuch der DO-28 konnten der Anstellwinkel $\alpha$, sowie der Trimmwinkel $\eta$ aufgezeichnet werden. Zur Darstellung wurden, nach Selektion der 8 Flugabschnitte, die Werte von $\alpha$ aufgrund der Trimmung $\eta$ aufgetragen. Dabei ist in dem Graph \ref{fig:alpha_eta_DO28} gut zu erkennen, dass die Wertepaare in guter Näherung in einer Flucht liegen. Zur Veranschaulichung wurde dazu noch eine Regression geplottet, die diesen ersten Eindruck untermauert.  

Es ergibt sich eine negative Steigung der Regressions-Gerade von $1.4$. Bei der Trimmung von $0 \ ^{\circ}$ ergibt sich aus der Regression ein entstehender Anstellwinkel $\alpha$ von $3.7 \ ^{\circ}$. So folgt daraus, dass bei einer Trimmung von $2.7 \ ^{\circ}$ der Anstellwinkel Null beträgt. Im Vergleich zu der Theorie zeigt sich ein sehr realistischer Verlauf der Trimmkurve, bei der es keine nennenswerten Abweichungen der Messpunkte von der Regressions-Geraden gibt.

\section{Auftriebsbeiwert über Anstellwinkel}

Im Plot \ref{fig:CA_alpha_DO28}, in dem $C_A$ über $\alpha$ aufgeführt ist, kann ebenfalls ein verhalten der Messpunkte bilanziert werden, wie bei der Trimmkurve. Die Messpunkte weisen ein linear steigendes Verhalten auf, was mit der Theorie übereinstimmend ist. Dazu ist auch hier eine lineare Regression durchgeführt worden, um dies zu stützen. Dabei stellt sich eine Gerade ein, deren Steigung bei $0.0772$ liegt. Dieser Wert stellt gleichzeitig das Derivat $C_{A\alpha}$ bzw. den Auftriebsanstieg dar. Infolge der Regression kann der Auftriebsbeiwert bei einem Anstellwinkel von $0 \ ^{\circ}$ als $C_{A0}$ sowie der Nullauftriebswinkel $\alpha_0$ mit $0.22$ bzw. $-2.89$ bestimmt werden.

Der Graph spiegelt somit den allgemeingültige Zusammenhang zwischen Anstellwinkel und Auftriebsbeiwert wieder. Dabei hat $\alpha$ per Definition und nach Abgleich mit dem Graphen einen Haupteinfluss auf den Auftriebsbeiwert. Dabei stellt sich für kleine Anstellwinkel wie erhofft ein linearer Verlauf ein \cite{Skript}.

\section{Lilienthal-Polare}
tbd

\section{Widerstand über Fluggeschwindigkeit}
tbd

\section{Staudruck über Anstellwinkel}
tbd

\section{Fluggeschwindigkeit über Anstellwinkel}
tbd